 %% This is file `elsarticle-template-1-num.tex',
%%
%% Copyright 2009 Elsevier Ltd
%%
%% This file is part of the 'Elsarticle Bundle'.
%% ---------------------------------------------
%%
%% It may be distributed under the conditions of the LaTeX Project Public
%% License, either version 1.2 of this license or (at your option) any
%% later version.  The latest version of this license is in
%%    http://www.latex-project.org/lppl.txt
%% and version 1.2 or later is part of all distributions of LaTeX
%% version 1999/12/01 or later.
%%
%% Template article for Elsevier's document class `elsarticle'
%% with numbered style bibliographic references
%%
%% $Id: elsarticle-template-1-num.tex 149 2009-10-08 05:01:15Z rishi $
%% $URL: http://lenova.river-valley.com/svn/elsbst/trunk/elsarticle-template-1-num.tex $
%%
%\documentclass[preprint,12pt]{elsarticle}
\documentclass[12pt]{iopart}
%% Use the option review to obtain double line spacing
%% \documentclass[preprint,review,12pt]{elsarticle}

%% Use the options 1p,twocolumn; 3p; 3p,twocolumn; 5p; or 5p,twocolumn
%% for a journal layout:
%% \documentclass[final,1p,times]{elsarticle}
%% \documentclass[final,1p,times,twocolumn]{elsarticle}
%% \documentclass[final,3p,times]{elsarticle}
%% \documentclass[final,3p,times,twocolumn]{elsarticle}
%% \documentclass[final,5p,times]{elsarticle}
%% \documentclass[final,5p,times,twocolumn]{elsarticle}

%% The graphicx package provides the includegraphics command.
\usepackage{graphicx}
%% The amssymb package provides various useful mathematical symbols
\usepackage{amssymb}
\usepackage{float}
\usepackage{amsmath}
\usepackage{hyperref}
\hypersetup{
    colorlinks=true,
    linkcolor=blue,
    filecolor=magenta,      
    urlcolor=cyan,
}
%% The amsthm package provides extended theorem environments
%% \usepackage{amsthm}

%% The lineno packages adds line numbers. Start line numbering with
%% \begin{linenumbers}, end it with \end{linenumbers}. Or switch it on
%% for the whole article with \linenumbers after \end{frontmatter}.
\usepackage{lineno}
\usepackage{caption}
\captionsetup[figure]{font=small}
%\usepackage[font=small,labelfont=bf]{caption}
%\usepackage[font=small,labelfont=bf]{subcaption}
\usepackage{subcaption}
\usepackage{graphicx}
\usepackage[%  
    colorlinks=true,
    pdfborder={0 0 0},
    linkcolor=red
]{hyperref}
%% natbib.sty is loaded by default. However, natbib options can be
%% provided with \biboptions{...} command. Following options are
%% valid:



%%   round  -  round parentheses are used (default)
%%   square -  square brackets are used   [option]
%%   curly  -  curly braces are used      {option}
%%   angle  -  angle brackets are used    <option>
%%   semicolon  -  multiple citations separated by semi-colon
%%   colon  - same as semicolon, an earlier confusion
%%   comma  -  separated by comma
%%   numbers-  selects numerical citations
%%   super  -  numerical citations as superscripts
%%   sort   -  sorts multiple citations according to order in ref. list
%%   sort&compress   -  like sort, but also compresses numerical citations
%%   compress - compresses without sorting
%%
%% \biboptions{comma,round}

% \biboptions{}
\begin{document}


%\journal{Journal Name}
%\tableofcontents

%\begin{frontmatter}

%% Title, authors and addresses

\title{Scattered light control at Ligo Livingston during third Observing run.}

%% use the tnoteref command within \title for footnotes;
%% use the tnotetext command for the associated footnote;
%% use the fnref command within \author or \address for footnotes;
%% use the fntext command for the associated footnote;
%% use the corref command within \author for corresponding author footnotes;
%% use the cortext command for the associated footnote;
%% use the ead command for the email address,
%% and the form \ead[url] for the home page:
%%
%% \title{Title\tnoteref{label1}}
%% \tnotetext[label1]{}
%% \author{Name\corref{cor1}\fnref{label2}}
%% \ead{email address}
%% \ead[url]{home page}
%% \fntext[label2]{}
%% \cortext[cor1]{}
%% \address{Address\fnref{label3}}
%% \fntext[label3]{}


%% use optional labels to link authors explicitly to addresses:
%% \author[label1,label2]{<author name>}
%% \address[label1]{<address>}
%% \address[label2]{<address>}

%\author{Siddharth Soni}

%\address{Louisiana State University, Louisiana, USA.}
\author{Siddharth Soni$^1$, Corey Austin$^1$,Anamaria Effler$^2$,Gabriela Gonzalez$^1$,Robert Schofield$^3$,Alex Urban$^1$,Guillermo Valdes$^1$}\address{$^1$ Department of Physics and Astronomy, Louisiana State University,Baton Rouge, LA, 70803,USA}\address{$^2$ LIGO Livingston Observatory, Livingston, LA 70754, USA}\address{$^3$ Department of Physics, University of Oregon, Eugene, OR 97403, USA}\ead{ssoni3@lsu.edu}
\begin{abstract}
%% Text of abstract

Noise due to scattered light has been one of the most frequent disturbances in the aLigo detectors. The scattering noise can be recognized as arches in the time-frequency plane. In this paper, we explore the scattering noise at Ligo Livingston Observatory for the third Observing run. We find two different populations of scattering noise and we investigate thoroughly the multiple origins of one of the population as well as its mitigation. Noise due to light scattering can interfere with the detection of gravitational waves. It is thus important to not only reduce the amount of scattering but also to identify the times at which the noise is present. To do that, we use data from sensors that monitor light transmitted through the end test masses of the detector.


\end{abstract}

%\begin{keyword}
%Science \sep Publication %\sep Complicated
%%% keywords here, in the %form: keyword \sep %keyword
%
%%% MSC codes here, in the %form: \MSC code \sep code
%%% or \MSC[2008] code %\sep code (2000 is the %default)
%
%\end{keyword}

%\end{frontmatter}

%%
%% Start line numbering here if you want
%%
%\linenumbers

%% main text
\section{Introduction}\label{introduction}
%\label{S:1}

%Maecenas \cite{Smith:2012qr}% %fermentum \cite{Smith:2013jd}% 
The Ligo  gravitational-wave observatories located at Hanford, Washington, and Livingston, Louisiana in the USA along with Virgo detector in Cascina, Italy~\cite{virgo} and GEO 600 detector in Germany~\cite{german} are a part of a worldwide network of Gravitational-wave detectors~\cite{aligo}. 
A schematic of the detectors is shown in Fig.~\ref{fig:schematic}.
Each dual recycled LIGO detector consists of a pair of 4 km long Fabry Perot cavities with returning beam combined in a Michelson interferometer. The detector act as a transducer for strain, converting the differential in the length of arms, into a signal that can be measured on a photo-diode. To increase the sensitivity required to detect gravitational waves, the detectors employ power recycling cavity and signal recycling cavity~\cite{signalr}. At the anti-symmetric end, the output mode cleaner filters the laser beam by blocking higher-order modes. Homodyne readout technique is used to read the signal at the output port \cite{Fricke_2012}. Finally, the output signal at the photodetectors is calibrated to an equivalent strain signal h(t) \cite{Abbott_2017,Viets_2018}.
%{Abbott_2017}\cite{Viets_2018}.
\begin{figure}[h]
    \centering
    \includegraphics[width=12cm]{ligo-schematic.png}
    \caption{A schematic of aLigo detectors taken from {~\cite{schematic}} }
    \label{fig:schematic}
\end{figure}


The first two observing runs, in September 2015-January 2016 and November 2016-August 2017, resulted in spectacular discoveries, including signals from the merger of 10 pairs of black holes and one from a merger of neutrons. stars~\cite{catalogue,firstpaper,neturonpaper}.

The ongoing, third observing run, began on April 1, 2019. On average, O3 has witnessed 1 detection per week. The BNS range at Livingston and Hanford, during O3, has been close to 130 Mpc and 110 Mpc respectively. An increase in the laser input power and adding a squeezer at the output, has been chiefly responsible for the increase in range from O2 to O3~\cite{src,prospect}.


Noise from several different sources limits the sensitivity of the strain data at different frequencies. 
While Shot noise is dominant at frequencies above 300 Hz~\cite{tjthesis}, ground motion is the major source of noise below 10 Hz which can get upconverted and affect the higher frequency band of the detectors. In addition to the stationary noise that includes quantum noise, suspension noise and seismic noise, short duration noise transients also known as glitches can affect the sensitivity of the detectors \cite{transientpaper}. These noise transients can be due to high ground motion, electronic malfunction or other reasons, not well-understood \cite{noisepaper,Cabero_2019}.

In this paper, we focus on the type of noise called scattering which impacts the gravitational strain in 10 Hz to 60 Hz frequency band. Scattering glitches due to ground motion is one of the most common sources of noise in this frequency band. Identifying and reducing the amount of scattering is imperative to increase the sensitivity of the detectors. 
 
The paper is composed as follows. In Sec \ref{scatteringnoise}, we discuss scattering noise and two primary methods used for its identification. Scattering noise during O3 is addressed in Sec   \ref{scatino3}. In Sec \ref{m0r0_scattering} we provide a mathematical treatment of Slow scattering, a sub-population of the light scattering and discuss the source of the noise. In Sec \ref{rzero} we develop a method to mitigate the slow scattering noise discussed in Sec \ref{m0r0_scattering}. Section \ref{transmon_scattering} introduces another source of slow scattering observed during O3. In Sec \ref{transmonwitness} we discuss a method to identify scattering times. Finally we summarize the paper in Sec \ref{summary}. 


 
 
 
 
\section{Scattering Noise.}\label{scatteringnoise}
Tiny imperfections on the surfaces of mirrors in the interferometer cause a small amount of light to scatter out of the main beam. This scattered light can then reflect from surfaces such as the chamber walls that are not well seismically isolated, back to the test mass. Upon recombining with the main beam, the scattered light introduces noise in the gravitational wave signal. The severity of noise depends on the relative motion between the mirrors and the scattering surface.

The motion of the scatterer introduces an additional phase in the field reflected from its surface. Consider a small fraction $f$ of the total light power that gets scattered back to the field in the arms, from a scattering surface located behind the end test mass (ETM). This light will acquire an additional phase due to the path length modulation caused by the relative motion between the end test mass and the scatterer. The resulting phase noise can be formulated as:
\begin{align}
    h_{scat} &= K\sqrt{f}\sin\left({\phi})\right \label{eq:1} \\
    {\phi}(t) &= \frac{4{\pi}}{{\lambda}}\left|{ x_{0} + {\delta} x_{sc}(t)}\right| \label{eq:2}
\end{align}
K is the coupling between motion of the scattering surface and strain while $\delta x_{sc}$ is the time-dependent displacement of the scattering surface which is added to the static path $x_{0}$~\cite{scatvirgo}. Scattering noise appears as an arch in time-frequency plane~\cite{omegscan}~\cite{omegscan1} as shown in  Fig.~\ref{fig:scat}.
\begin{figure}[h]
    \centering
    \includegraphics[width=10cm]{scatindarm.png}
    \caption{Scattering noise shows up as arches in the time-frequency spectrograms.}
    \label{fig:scat}
\end{figure}

This peak frequency tells us the number of fringes per second and can be related to the velocity of the scatterer, $v_{sc}$ with the following:
\begin{equation}
    f_{fringe}(t) = \left|\frac{2v_{sc}(t)}{\lambda}\right|  \label{eq:3}
\end{equation}
where $f_{fringe}(t)$ is the fringe frequency. We can also look at the spacing of the peaks in the time domain to give us the period of the scattering surface.

During the first and second Observing runs, scattering noise was identified by measuring the motion of optics that could be responsible for scattering and by looking at the output of GravitySpy which is a noise classification tool. Next, we briefly discuss these two methods.


\subsection{GravitySpy}
GravitySpy is an image recognition tool that uses machine learning to classify the variety of glitches that show up as transient noise in the strain data~\cite{gspy}~\cite{gspymachine}. It is a citizen science project and volunteers help to generate the training dataset by assigning one of the several glitch class to the spectrogram images. The algorithm is then trained, on the time-frequency spectrograms of these labeled glitches. Based on this model training, the algorithm then classifies a new glitch into one of the several transient noise categories, scattering being one of them.

\begin{figure}[h]
    \centering
    \includegraphics[width=10cm]{gspyglitch1.png}
    \caption{Time-frequency spectrograms for three different glitch classes, classified by GravitySpy. Blip glitches show up randomly in the data with no known mechanism associated. Scattering shows up during high ground motion, especially during earthquakes. Extremely Loud triggers are associated with very high SNR and large range drops. }
    \label{fig:gspyimages}
\end{figure}



\subsection{Monitoring motion of possible scattering mirrors.}\label{osems}
Another way of identifying the noise is by monitoring the motion of optics using optical sensors electronic modulators (OSEMSs) and then correlating this motion with the presence of glitches in the gravitational wave strain channel in 10 Hz - 60 Hz frequency band.

This is accomplished using an algorithm called - \textbf{gwdetchar-scattering}~\cite{alex_l_urban_2019_3526829}. The list of OSEMs we look for evidence of scattering in is not constant. This list is updated with channels getting added or removed based on how good a witness of scattering they are. For each sensor in this list, the gwdetchar-scattering code creates segments of the form (startime, endtime) during which the motion in optic crosses a certain frequency threshold.  The code then looks for time coincidences between these segments and glitches in h(t). Efficiency is defined as the percentage of triggers in h(t) channel that is captured by these segments while the deadtime is the duration of all the segments for an optic as a percentage of total observing time. Scattering is considered to be ``strong" if the ratio of efficiency over deadtime is greater than 2 and ``weak" if it is less than 2. 

\subsection{Comparing the two methods}
GravitySpy, even though does not provide any information with regards to where the scatterer might be located, identifies a larger subset of scattering triggers compared to that identified by motion in OSEMs. On the other hand, optics motion as discussed previously can be a more direct method of locating the source of scattering noise since it can identify which mirror is moving with the velocity required. It thus makes more sense to see if we can increase the overall ``Efficiency'' of \textbf{gwdetchar-scattering} by adding better scattering witnesses to the algorithm. Later in section \ref{transmonwitness}, we explore such a scattering witness that can be used to identify the scattering noise. A method based on Hilbert Huang transform has also been developed to catch scattering surfaces \cite{hhtransform}. 

\quad


\section{Scattering in O3.}\label{scatino3}
The BNS range of the detectors, which is the average astrophysical range for binary star mergers, increased to 110-140 Mpc in O3, from around 80-100 Mpc in O2.~\cite{prospect} Due to this improvement in sensitivity, the same noise now surfaces with higher SNR.
Consequently, we see a lot more scattering arches in O3. An interesting feature of scattering in O3 is the presence of two different populations of scattering glitches, so-called the slow scattering and the fast scattering. The glitch morphology of the slow scattering is the more familiar arch in time-frequency plane which differs from the fast scattering trigger which is more localized in time. GravitySpy currently does not distinguish between fast and slow scattering. Slow arches during O3a  typically appeared with an SNR of 15 or higher. SNR in the range of 14-16 serves as a decent threshold to separate the Slow and Fast scattering glitches. In this analysis, we have used the SNR threshold of 15 to separate the two populations of scattering glitches during O3a.

\begin{figure}[h]
   \centering
    \begin{subfigure}[b]{0.45\textwidth}
        \centering
         \includegraphics[width= \textwidth]{slowarch.png}
         \caption{Slow scattering triggers at 1238512902.}
         \label{fig:slowscat}
    \end{subfigure}
    \hfill
    \begin{subfigure}[b]{0.5\textwidth}
        \centering
         \includegraphics[width =\textwidth,height=4cm]{fastscat3.png}
         \caption{Fast scattering triggers at 1243433257.09.}
         \label{fig:fastscat}
         
    
    \end{subfigure}
    \caption{Slow and Fast scattering in O3a.}
    \label{fig:fastslowscat}
    
\end{figure}

\subsection{Slow Scattering.}
In time-frequency spectrograms, slow scattering triggers resemble an arch.  Fig.\ref{fig:slowscat}. shows slow scattering in h(t) for a day with large ground motion.  As can be observed from this figure, the arches have a period of 7.3 seconds, giving a scattering surface frequency close to 0.13 Hz~\cite{alogsid}. 
Slow scattering is more dominant during noise in $0.03$ - $0.1$ Hz (earthquake) and $0.1$ - $0.3$  Hz (microseism) band. During O3, it was particularly strong on Dec 1, 2019 and January 6, 2020.  Depending on ground motion, this noise creates scatter shelves in the frequency band 10 Hz to 120 Hz as shown in Fig.\ref{fig:scatshelf1}.

\begin{figure}[h]
    \centering
    \includegraphics[width=10cm]{scatshelf1.png}
    \caption{ Depending on the ground motion, scattering create shelves in h(t) spectrum that are visible in the range 10 Hz to 120 Hz. The blue curve is from January 6, 2020, a day with extremely high microseism compared against a quieter day.}
    \label{fig:scatshelf1}
\end{figure}



\subsection{Fast Scattering.}
The fast scattering triggers occur with a frequency close to 4 Hz~\cite{alogjosh}. This is in contrast to the 0.13-0.14 Hz of slow scattering arches. The noise  correlates better with the 1-3 Hz anthropogenic band and is usually higher in the daytime during the weekdays. June 3, 2019, a day with very high anthropogenic noise in the 1-3 Hz band registered close to 1000 triggers that resembled fast scattering. 


Table \ref{tab:fastslowtab} compares different characteristics of Slow and Fast Scattering.
\begin{table}[h]
    \centering
    \begin{tabular}{c|c|c}
    \hline
      \textbf{Properties}   &  \textbf{Slow Scattering} & 
      \textbf{Fast Scattering}\\
      \hline
       Frequency of arches  & $<$ 0.2 Hz & $>$ 3 Hz \\
      \hline
      Median Peak Frequency & 20.5 Hz & 32.3 Hz \\
      \hline
      Median duration & 3.4 sec & 1.3 sec \\
      \hline
      Fraction of Scattering & 32.1 \% & 67.9 \% \\
       
    \hline
    \end{tabular}
    \caption{Comparison of Slow and Fast scattering triggers in O3a.}
    \label{tab:fastslowtab}
\end{table}{}


The striking differences in the glitch morphology, the peak frequency and the duration of the triggers for slow and fast scattering triggers suggest that they have a different mechanism through which the noise couples to the detector. Fig.~\ref{fig:snrdur}. shows the SNR and duration of total scattering glitches in O3. Both the distribution reveals the presence of more than one population of scattering triggers. As evident from the Table.~\ref{tab:fastslowtab} slow scattering triggers tend to be long duration compared to fast scattering triggers. We will not investigate fast scattering further in this paper. In the next section, we provide a detailed description of slow scattering noise at Livingston.
\begin{figure}[h]
   \centering
    \begin{subfigure}[b]{0.45\textwidth}
        \centering
         \includegraphics[width= 6cm]{histsnr2.png}
         \caption{SNR distribution for scattering in O3.}
         \label{fig:histsnr}
    \end{subfigure}
    \hfill
    \begin{subfigure}[b]{0.5\textwidth}
        \centering
         \includegraphics[width =6cm]{histdur2.png}
         \caption{{\small Duration distribution of scattering in O3.}}
         \label{fig:histdur}
         
    
    \end{subfigure}
    \caption{The SNR and duration histograms for total scattering in O3 reveal the presence of two populations the Fast and Slow scattering. Slow scattering tends to be louder and long duration. During notably high micro-seismic motion, there are several more slow scattering arches, that causes the third bump in the duration plot.}
    \label{fig:snrdur}
    
\end{figure}



\section{Slow Scattering noise at LLO}\label{m0r0_scattering}
Slow scattering noise pollutes the h(t) spectra during high ground motion in $0.1$-$0.3$ Hz band. During operations, an external drive is sent to the reaction chain to move the test mass chain to keep the detector on resonance. This creates relative motion between the end test mass (ETM) and other objects in its vicinity. 
Due to this relative motion, any light trapped between the test mass and one of these objects can potentially introduce phase noise once it rejoins the main beam in the detector arms. 

\begin{figure}[h]
    \centering
    \includegraphics[width=10cm]{mirror-diag1.png}
    \caption{During high ground motion, the reaction chain pushes on the main chain  to keep the detector on resonance. The control is applied via coil actuators and electrostatic drive as shown here. This fluctuates the distance between the two chains. A part of the light reflected back from the gold electrostatic drive joins the main beam in the arm with an additional phase. This difference in phase of the two beams shows up as scattering noise.}
    \label{fig:mirror_diag}
\end{figure}

Consider specifically the relative motion between the end test mass and the end reaction mass (ERM).
When a part of the light between these two mirrors is transmitted back to the arms, it interferes with the main beam. The difference in phase of the two light beams introduces phase noise in the gravitational wave channel. The information of the path length modulation between the ETM and the ERM is stored in the phase of the light between the two mirrors given by equation 3.

In this case, ${\delta x_{sc}}$ is the relative movement between the ETM and ERM. Below the natural pendulum frequencies, it is similar to the relative movement between the penultimate stage (L2) mirrors on test mass chain (M0) and reaction chain (R0).  This allows us approximate the relative motion at the end stage by L2 stage motion.


As shown in Fig.~\ref{fig:fringel2}. the fringe frequency as given by equation 2, of this L2 stage motion and its higher harmonics matches the multiple scattering arches in h(t) spectrograms pretty well. The multiple harmonics of the scattering arches suggests that light is reflected multiple times between the test masses.  For these multiple reflections, we expect the total power reflecting back in the scattered field to decrease. 
\par
The ERM is in the shape of a hollow cylinder. Most of the light, transmitted through the ETM, goes through the hole onto the other side of the reaction chain as shown in Fig.~\ref{fig:mirror_diag}. A small fraction of this Gaussian beam hits the gold trace electrostatic drive (ESD) on the ERM. Due to its high reflectivity, almost all of it is scattered back towards the test mass and a fraction $\sqrt{T_{end}}$ gets transmitted back to the arm through the ETM. This back scattered field with an additional phase shift, given by equation \ref{eq:2}, interferes with the main beam in the arms and introduces phase noise in h(t).
\par
Let $E_{0}$ be the field in the arms and $E_{asd}$ is the part of this field backscattered from the end reaction mass at the point $E_{0}$ is computed. We can calculate the total field in the arms:

\begin{align}
    E_{tot} &= E_{0} + E_{asd} \label{eq:4} \\
    E_{asd} &= E_{0}Ae^{i{\delta}{\phi}(t)} ,\quad A = T_{end}\sqrt{f_{r}}  \label{eq:5} \\ 
    E_{tot} &= E_{0}[1 + Ae^{i{\delta}{\phi}(t)}]  \label{eq:6}
    %E_{tot} &= E_{0}[1 + Acos({\delta}{\phi}(t)) + Asin({\delta}{\phi}(t))]
\end{align}


The equivalent strain for this phase noise is:
\begin{align}
    h_{sc} &= \frac{A{\lambda}}{4{\pi}L}sin{\delta}{\phi}  \label{eq:7}
\end{align}

The effect on sensitivity can now be written as:
\begin{equation}
    h(f) = \frac{\lambda T_{end}\sqrt{f_{r}}}{8{\pi}L}\mathcal{F}[sin(\frac{4{\pi}}{\lambda}{\delta x_{sc}(t)})]  \label{eq:8}
\end{equation}

$T_{end}$ is the end test mass transmission, L is the length of arms $f_{r}$ is the fraction of power reflected back from the scattering surface, and $\mathcal{F}$ is the Fourier transform of the scatterer motion \cite{vaj_scat}.
\begin{figure}[h]
   \centering
    \begin{subfigure}[b]{0.45\textwidth}
        \centering
         \includegraphics[width= \textwidth,height=4.7cm]{fringel2.png}
         \caption{Fringe frequency L2 stage motion.}
         \label{fig:fringel2}
    \end{subfigure}
    \hfill
    \begin{subfigure}[b]{0.45\textwidth}
        \centering
         \includegraphics[width =\textwidth]{spectruml2a.png}
         \caption{L2 stage asd overlaid on h(t) spectrum.}
         \label{fig:spectruml2}
         
    
    \end{subfigure}
    \caption{In the first plot, we have overlaid the multiple harmonics of the fringe frequency due to the penultimate (L2) stage motion onto scattering arches. In the second plot, we have plotted the DARM spectrum and the spectrum calculated from the L2 stage motion using the equation 3 with smaller and smaller values for $f_{r}$ as we match higher frequency shelves.}
    \label{fig:fringespectrum}
    
\end{figure}

The $\delta x_{sc} $ in equation 8 is the relative displacement between the test mass chain and the reaction chain which is measured by the optical sensors on the L2 stage. 

The first shelf in the h(t) spectrum in Fig.~\ref{fig:spectruml2}. matches for $\sqrt{f_{r}} = 4e^{-4}$, an order of magnitude smaller than Hiro's estimate\cite{hiro_calc}. The coefficient for second and third shelf are $9e^{-5}$ and $9e^{-6}$ respectively, about an order of magnitude reduction for each higher harmonic.

\qquad



\section{Noise mitigation using suspensions control system.} \label{rzero}
\quad
\begin{figure}[h]
    \centering
    \includegraphics[width=5cm]{R0trackingdiagram.png}
    \caption{A part of the drive from L2 stage is sent to the R0 stage. This decreases the relative motion between the lower stages of main and reaction chain, thereby reducing the scattered light coupling to the main beam in the detector arms.}
    \label{fig:r0tracking}
\end{figure}
In order to keep the mirrors on resonance during high microseism, an external drive is sent to the coil/magnet actuators on the upper three masses of the reaction chain. These actuators push on the test mass chain creating relative motion between the two chains. The distance fluctuation is different for different stages of the quad suspension. At low frequencies, the relative motion between the ETM and the ERM is similar to that between the L2 stage of M0 and R0 chains. During extremely bad microseism as witnessed on Jan 6, 2020, this relative motion between the ETM and ERM can be as high as 25${\mu}$m/s. For any ${\delta}x >> {\lambda}$ there is an upconversion of the low-frequency motion of the mirrors which shows up as scattering shelves/arches with the frequency determined by Eq 2. In h(t) spectrograms, scattering arches in the frequency band 10 Hz to 120 Hz have been observed due to light trapped between the ETM and ERM during high microseism.

One way to reduce this noise coupling is by reducing the relative motion at the lower stages of the test mass and reaction mass chains while keeping the intended relative motion between the ETM and the input test mass (ITM) in the arm cavity. This can be achieved by sending a part of the drive from the L2 stage and feeding it to the R0 stage as shown in Fig.~\ref{fig:r0tracking}. This will cause the two chains to move together and hence reduces the relative motion between the M0 and R0 chains. The reaction chain ``tracks'' the main chain~\cite{alog_robert}. Fig.~\ref{fig:l2_comparison}. compares the distance fluctuations between the L2 stage of M0 chain and L2 stage of the R0 chain before and after the tracking was implemented, for two days with similar levels of microseism.

\begin{figure}[h]
    \centering
    \includegraphics[width=12cm]{feb7_dec22l2.png}
    \caption{R0 tracking sends a part of the drive from the L2 stage to R0 stage, effectively reducing the relative motion between the test mass and reaction mass chains at the lower stages. L2 stage BOSEM measures the distance fluctuation between the L2 stages of test mass and reaction mass chain. On Feb 7, 2020, a day with similar level of microseism as Dec 22, 2019, the relative motion between the two chains have been greatly reduced.}
    \label{fig:l2_comparison}
\end{figure}

\subsection{M0-R0 Scattering Post R0 tracking.}
R0 tracking was implemented on Jan 7, 2020, at LLO~\cite{alog_anamaria}. To understand the impact of the tracking on slow scattering caused by M0-R0 relative motion, we observed the SNR of scattering triggers and ground motion in the earthquake and microseism band between Nov 1, 2019, the start of O3b and Feb 8, 2020. These triggers are classified as scattering by GravitySpy with confidence above 0.9. We divided this data into Pre and Post R0, where Pre R0 is from Nov 1, 2019, to Jan 6, 2020, and Post R0 is from Jan 10, 2020, to Feb 8, 2020. The analyzed data is normalized by the observing duration of Post R0 considered in this study ($\sim 21$ days). We found a clear reduction in the SNR of the scattering triggers for the Post R0 scattering without any considerable change in the ground motion~\cite{alogsid_R0}. The results of this analysis are shown in Fig.~\ref{fig:snr_r0}.
\par
\begin{figure}[h]
   \centering
    \begin{subfigure}[b]{0.45\textwidth}
        \centering
         \includegraphics[width= \textwidth]{snr_r0.png}
         \caption{SNR comparison Pre and Post R0 tracking.}
         \label{fig:snr_r0}
    \end{subfigure}
    \hfill
    \begin{subfigure}[b]{0.45\textwidth}
        \centering
         \includegraphics[width =\textwidth]{rate_r0.png}
         \caption{Glitch rate comparison for O3a, Pre and Post R0 tracking in O3b.}
         \label{fig:rate_r0}
         
    
    \end{subfigure}
    \caption{The first image compares the SNR of Scattering glitches Pre and Post R0 tracking. The levels of microseism were found to be similar for the observing duration used in this comparison. This shows a clear reduction in the SNR of the scattering triggers following R0 tracking implementation. The second image compares the rate of scattering triggers for different levels of microseism. This shows that for similar levels of microseism above $1{\mu}m/s$, the Post R0 glitch rate (in green) is substantially lower than Pre R0 (in orange).}
    \label{fig:postr0_scat}
    
\end{figure}
We also compared the rate of scattering triggers against the microseismic ground motion for O3a and O3b, with O3b divided into Pre and Post R0. Here again, we found that for similar levels of microseism above 1${\mu}m/s$, the Post R0 glitch rates are considerably lower as shown in Fig.~\ref{fig:rate_r0}.~\cite{alog_corey}.



\section{Relative motion between test mass and transmitted light monitors.} \label{transmon_scattering}
As mentioned in Section \ref{rzero} , the control drive sent to the reaction chain creates relative motion between the test mass chain and other objects in its vicinity. Relative motion between the test mass and the monitors of transmitted light is found to be another source of scattered light in DARM.
\par
\begin{figure}[h]
    \centering
    \includegraphics[width=12cm]{mirror-diag2.png}
    \caption{During high ground motion, the magnetic actuators on the top three masses of reaction chain and electrostatic drive on the end reaction mass push the test mass chain to keep the interferometer on resonance. This changes the distance between the test mass and other objects in its vicinity. M0-R0 scattering discussed earlier correlates with $L_{1}$ fluctuations. TMS or transmission motor system behind the mirrors, houses quadrant photo-diodes. These photo-diodes are used to check for the alignment of the cavity. Distance fluctuations between these monitors of  transmitted light and the test mass causes ETM-TMS scattering.}
    \label{fig:my_label}
\end{figure}

A small fraction of light that gets transmitted through the end test mass, is received by the quadrant photo-diodes on the Transmission motor system (TMS) behind the Internal Seismic Isolation cage. During high microseism, the path length between the ETM and the TMS is modulated, effectively changing the phase of the light trapped between them. This phase noise then shows up as scattering arches in both h(t) and the transmitted photo-diode channel at the same frequency. 
Since the dominant motion of ETM with respect to nearby objects is the DARM drive in between the quad chains, the motion between the M0 chain and the R0 chain is twice the motion of the ETM (with respect to other objects like the cage, walls or TMS). Thus, the scattering shelf/arch due to ETM-TMS relative motion is observed at one-half the frequency of the scattering shelf/arch due to M0-R0 relative motion. This can be seen in the Fig.~\ref{fig:1262darm}. where the first harmonic due to M0-R0 scattering is at 40 Hz and the scattering arch due to ETM-TMS noise coupling is at 20 Hz.

\begin{figure}[h]
   \centering
    \begin{subfigure}[b]{0.45\textwidth}
        \centering
         \includegraphics[width= \textwidth]{1262389546darm.png}
         \caption{Scattering arches in h(t).}
         \label{fig:1262darm}
    \end{subfigure}
    \hfill
    \begin{subfigure}[b]{0.45\textwidth}
        \centering
         \includegraphics[width =\textwidth]{1262389546trans.png}
         \caption{Scattering arch in X end transmmited photo-diode.}
         \label{fig:1262trans}
         
    
    \end{subfigure}
    \caption{The first figure shows scattering arches in h(t) during a day with very high microseism. Multiple reflections between the test mass and reaction mass generates the multiple harmonics. The first harmonic of the light scattering due to M0-R0 relative motion is close to 40 Hz with higher harmonics present at $\sim$ 80 Hz, $\sim$ 120 Hz. Since the relative motion between the test mass and TMS is one-half of the M0-R0 motion, the scattering arch due to ETM-TMS scattering is at $\sim$ 20 Hz. The second plot shows the scattering arch in the transmitted photo-diode channel at $\sim$ 20 Hz.}
    \label{fig:1262scat}
    
\end{figure}
The presence of scattering arches in these transmitted light monitors serves as a witness of scattering in h(t). Whenever there is relative motion between M0 and R0 chains, the distance between the ETM and TMS also fluctuates. These are two separate but simultaneous path length fluctuations with a common source, the test mass motion. Scattering arches due to both of these distinct path length fluctuations are thus present at the same time in h(t) albeit at different frequencies.
\par
R0 tracking does not affect the relative movement of the test mass chain with respect to the ISI cage or TMS or chamber walls. The light scattering due to ETM-TMS coupling thus remains unchanged. The scattering arch/shelf due to this noise coupling is very rarely above 10 Hz in both h(t) and transmitted photo-diode channel. In order to make a shelf at $f$ Hz, the relative velocity must be at least $\frac{f{\lambda}}{2}$, so for 10 Hz, we need the relative motion between the ETM and TMS to be at least $5\mu{\rm m/s}$. This greater than 5-micron relative motion occurs during high ground motion in $0.1$-$0.3$ Hz band (above $2\mu{\rm m/s}$) coupled with high anthropogenic noise~\cite{alogsid_trans}. Fig.~\ref{fig:etm-transmon}. shows scattering shelf in h(t) and X end transmitted photo-diode channel due to ETM-TMS coupling during a period of high ground motion in the microseism ($0.1$-$0.3$ Hz) band.
\par
Prior to R0 tracking, a scattering shelf in transmitted photo-diode channel at $f$ Hz will predict scattering shelves in h(t) at $f$ Hz due to ETM-TMS coupling and at 2$f$, 4$f$, 6$f$ and so on due to M0-R0 coupling. Following R0 tracking, a shelf in transmitted photo-diode channel at $f$ Hz only corresponds to a shelf in h(t) at the same frequency. The higher frequency scattering shelves due to M0-R0 coupling have gone away due to reduced coupling. Fig.~\ref{fig:etm-transmon}.
compares the scattering shelves in h(t) and X end transmitted photo-diode channel for scattering events before and after R0 tracking.
\par
During O3, the control drive was sent to the X end of the detector. One possible remedy to reduce the ETM-TMS noise coupling is to split and apply this drive at both ends. Further reduction can be employed by making the TMS follow the test mass chain. 
\begin{figure}[h]
    \centering
    \includegraphics[width=\textwidth]{etm-transmon.png}
    \caption{The left image shows scattering shelves in X end transmitted photo-diode and h(t) before R0 tracking was implemented. The scattering shelf at 10 Hz in h(t) correlates with ETM-TMS noise coupling, while the higher frequency shelves at $\sim$ 20 Hz and $\sim$ 40 Hz are due to M0-R0 scattering. The image on the right is for a Post R0 tracking scattering event and higher frequency shelves are absent in h(t) due to reduced relative motion between the main chain and the reaction chain.}
    \label{fig:etm-transmon}
\end{figure}

 
\section{Using transmitted photo-diodes to identify scattering.}\label{transmonwitness}
In last section, we showed that transmitted photo-diode channels at the drive end serve as a witness of scattering noise in h(t). Fig.~\ref{fig:htxtrbo3}. shows a time correlation between the scattering triggers in h(t) and the noise in the photo-diode channel. Due to the presence of this temporal coincidence of triggers, the noise in the transmitted photo-diode channel can be used to identify the scattering noise in h(t). Along with OSEM’s mentioned in section \ref{osems}, these channels can be added to the list of scattering witnesses in the algorithm - \textbf{gwdetchar-scattering}.

Apart from slow scattering, Extremely Loud triggers, which is another class of triggers characterized by typically high SNR also makes its way to the transmitted photo-diode channels. To differentiate the presence of slow scattering noise in h(t) from loud glitches, we can look at the frequency content of the coincident noise in the transmitted photo-diodes. The scattering noise in these witness channels appears in the range  4 - 10 Hz while the triggers coincident with loud glitches in h(t) typically appear with higher peak frequency. We can use this fact to isolate the presence of slow scattering in h(t).
\subsection{Band-limited RMS}
To capture the scattering triggers in h(t), we use whitened band-limited RMS segments constructed from the raw time-series of the transmitted photo-diode channels, in the frequency band of interest. A scattering trigger in h(t) shows up as a spike in these band-limited RMS also known as blrms. By choosing a suitable threshold, we can create blrms segments and then use time coincidence with the triggers in h(t) to identify the scattering triggers. Any h(t) triggers that coincide with these band-limited segments are then written to a file as potential scattering trigger times.  This process is shown in Fig.~\ref{fig:flow}. It is important to note that a glitch in the transmitted photo-diode channel outside the band of interest will not result in a spike and this ensures that blrms segments only catch scattering noise in h(t). Before finding time coincidence between the h(t) triggers and blrms segments, we filter the triggers by SNR (between 15 and 200) and frequency (between 10 Hz and 60 Hz). This is done to reduce the size of the target dataset and scattering noise occurs well within these bounds. This will also reduce the number of false positives. 




\begin{figure}[h]
   \centering
    \begin{subfigure}[b]{0.43\textwidth}
        \centering
         \includegraphics[width = \textwidth]{htxtrbo3.png}
         \caption{Noise correlation between X end transmitted photo-diode and slow scattering in h(t) channel during the first 12 weeks of O3a.} %Event rate in the transmitted photo-diode channel increases when slow scattering noise dominate in DARM.}
         %Transmon scattering segments can be generated from this, whenever blrms cross a well defined threshold.}
         \label{fig:htxtrbo3}
    \end{subfigure}
    \hfill
    \begin{subfigure}[b]{0.43\textwidth}
        \centering
         \includegraphics[width = \textwidth]{blrmsfringe2.png}
         \caption{Correlation between motion in L2 stage in ETMX and  noise in 4-10 Hz band in X end transmitted photo-diode during a period of high ground motion.}
         \label{fig:blrmsfringe2}
         
    
    \end{subfigure}
    \caption{ X end transmitted photo-diode as a witness of slow scattering noise in gravitational wave channel.}
    \label{fig:transetmx}
\end{figure}
\quad



\begin{figure}[h]
    \centering
    \includegraphics[width=10cm]{flow.png}
    \caption{Flowchart of the process to capture scattering in h(t) by using segments generated from whitened transmon blrms.}
    \label{fig:flow}
\end{figure}

\subsection{Results}


\par
We performed the analysis for O3a, from April 1, 2019, to Sep 30, 2019. The blrms segments identified 3864 h(t) triggers as scattering while GravitySpy found 3663 scattering triggers above confidence of 0.8.  Three-fourth of the GravitySpy scattering match with scattering triggers caught by blrms segments. While 71 $\%$ of the 3864 triggers caught by these blrms segments match with GravitySpy output. This suggests that 29 $\%$ of 3864 or approximately 1120 triggers are false positive with respect to GravitySpy. 
The time-frequency spectrograms of 57 randomly chosen trigger times from these 1120 triggers, showed that as many as 40 of these were scattering triggers but they were not identified as scattering by GravitySpy above confidence of 0.8. 




\begin{figure}[h]
   \centering
    \begin{subfigure}[b]{0.45\textwidth}
        \centering
         \includegraphics[width= \textwidth]{transmonvetoed.png}
         \caption{}
         \label{fig:transpie}
    \end{subfigure}
    \hfill
    \begin{subfigure}[b]{0.42\textwidth}
        \centering
         \includegraphics[width =\textwidth]{gspyvetoed.png}
         \caption{}
         \label{fig:gspypie}
         
    
    \end{subfigure}
    \caption{For O3a, the first pie chart shows that most of the triggers vetoed by  blrms segments are indeed scattering. The omega scans of the false positives show quite a few of them are scattering but were labeled with a confidence lower than 0.8 by GravitySpy. The second pie chart shows that blrms segments caught close to 75 \% of the slow scattering that GravitySpy finds above confidence of 0.8.}
    \label{fig:gspytrans}
    
\end{figure}
\par

\begin{figure}[h]
    \centering
    \includegraphics[width=10cm]{osemstrans.png}
    \caption{Comparing efficiency over deadtime between OSEMs and blrms segments. We can see that on all three days, transmmited photo-diode's band limited rms segments perform a better job than OSEMs in identifying slow scattering noise.}
    \label{fig:osemtrans}
\end{figure}
\vspace{2cm}
We also examined the performance of blrms segments against OSEMs on 3 days in O3a dominated by slow scattering noise. Fig.~\ref{fig:osemtrans}. shows the efficiency over dead time for Apr 13, 2019, May 13, 2019, and June 25, 2019. Efficiency is the percent of filtered DARM triggers that coincide with blrms segments. Deadtime refers to the total duration of the segments as a fraction of the total observing duration for that day. A large value of efficiency over deadtime is preferred as the goal is to lose as little time as possible with maximum efficiency. For all the three days, blrms segments capture more scattering noise than OSEM scattering segments.

\newpage
\section{Summary and discussion.}\label{summary}
 Scattering noise impairs the quality of data at  aLigo detectors during high ground motion. Upconversion of the low-frequency noise due to large optic motion reduces the sensitivity of the detector in the high-frequency detection band. It is thus imperative to control the amount of scattering and it is also important to identify when scattering is present in the gravitational strain data. 

We analyzed light scattering in LLO during O3 and we found the presence of two different populations of scattering noise, the Slow Scattering, and the Fast Scattering. We investigated slow scattering that appears with a typical arch shape in the time-frequency representation and we found two different paths through which this noise couples to the detector simultaneously. We were able to implement a solution for the louder noise coupling that resulted in a substantial reduction of the noise and we discussed possible remedies for the other one. 

In order to identify the times when this noise is present in the gravitational wave channel, we suggested using the excitations in the band-limited time-series data of an auxiliary channel. This channel, monitors the light transmitted through the end test mass and serves as a scattering witness.


  
%% The Appendices part is started with the command \appendix;
%% appendix sections are then done as normal sections
%% \appendix

%% \section{}
%% \label{}

%% References
%%
%% Following citation commands can be used in the body text:
%% Usage of \cite is as follows:
%%   \cite{key}          ==>>  [#]
%%   \cite[chap. 2]{key} ==>>  [#, chap. 2]
%%   \citet{key}         ==>>  Author [#]

%% References with bibTeX database:

%\bibliographystyle{model1-num-names}
%\bibliography{sample.bib}

%% Authors are advised to submit their bibtex database files. They are
%% requested to list a bibtex style file in the manuscript if they do
%% not want to use model1-num-names.bst.

%% References without bibTeX database:

\newpage
\section*{References}
\bibliographystyle{unsrt}
\bibliography{sample.bib}
%\begin{thebibliography}{00}

%% \bibitem must have the following form:
%%   \bibitem{key}...
%%

%\bibitem{1}

%\end{thebibliography}
%\section{Referencess}
%\begin{itemize}
%    \item [1] Advanced Ligo LIGO-P1400177-v5
%    \item [2] Acernese F et al. (Virgo Collaboration) 015 %Class. Quant. Grav. 32 024001 (Preprint %arXiv:1408.3978)
%    \item [3]  Dooley K L et al. 2016 Class. Quant. Grav. %33 075009 (Preprint arXiv:1510.00317) URL %https://doi.org/10.1088/0264- 9381/33/7/075009
%    \item [4] AkutsuTetal.(KAGRA)2017ThestatusofKAGRAundeg%roundcryogenicgravitationalwave telescope 15th %International Conference on Topics in Astroparticle %and nderground Physics (TAUP 2017) Sudbury, Ontario, %Canada, uly 24-28, 2017 (Preprint arXiv:1710.04823)
%    
%\item [5] B. J. Meers, Phys. Rev. D 38, 2317 (1988).
%\item [6] GWTC-1: A Gravitational-Wave Transient Catalog %of Compact Binary Mergers Observed by LIGO and Virgo %during the First and Second Observing Runs. %arXiv:1811.12907
%\item [7] On the Progentor of Neutron Star Merger %GW170817.
%arxiv.1710.05838
%\item [8] Prospects for Observing and Localizing %Gravitational-Wave Transients with Advanced LIGO, dvanced %%Virgo and KAGRA arXiv:1304.0670v9
%\item [9] Detector Characterization for Advanced Ligo, %Thesis Thomas J Massinger.
%\item [9] Noise from scattered light in Virgo's second %science run data. T Accadia et al 2010 Class. Quantum %Grav. 27 194011
%\item [10] https://alog.ligo-la.caltech.edu/aLOG/index.ph%%?callRep=45053
%\end{itemize}

\end{document}

%%
%% End of file `elsarticle-template-1-num.tex'.